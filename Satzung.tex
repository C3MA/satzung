\documentclass[a4paper, 12pt]{scrartcl}
\usepackage[utf8]{inputenc}
\usepackage[T1]{fontenc}
\usepackage[ngerman]{babel}
\pagestyle{plain}
\title{Satzung}
\subtitle{des Chaos Computer Club Mannheim}
\author{}
\date{04. April 2014}

\renewcommand*\thesection{\S~\arabic{section}}
\KOMAoptions{toc=flat}
\begin{document}
\maketitle
\sffamily
\tableofcontents

\newpage
\section{Name, Sitz, Geschäftsjahr}
\begin{enumerate}
	\item Der Verein führt den Namen "`Chaos Computer Club Mannheim"' und soll in das Vereinsregister eingetragen werden; nach der Eintragung führt er den Zusatz "`e.V."'.
	\item Der Verein hat seinen Sitz in Mannheim.
	\item Das Geschäftsjahr des Vereins ist das Kalenderjahr.
\end{enumerate}

\section{Zweck des Vereins}
\begin{enumerate}
	\item Zweck des Vereins ist die Förderung der Bildung und Volksbildung auf dem Gebiet der Informationstechnologien, des Informationsrechts und verwandten Themen sowie des kreativen Umgangs mit diesen.
	\item Der Verein verfolgt ausschließlich und unmittelbar gemeinnützige Zwecke im Sinne des Abschnittes "`Steuerbegünstigte Zwecke"' der Abgabenordnung.
	\item Der Satzungszweck wird insbesondere verwirklicht durch:
	\begin{enumerate}
		\item Durchführung von öffentlichen, entgeltfreien Veranstaltungen für Computersicherheit, Informationsrecht und kreativen Umgang mit neuen Technologien und deren Anwendungen.
		\item Förderung von Forschung, Entwicklung und Aufklärung im Bereich der Informationstechnologien.
		\item Förderung der Allgemeinbildung der Bevölkerung im Umgang mit neuen Technologien.
	\end{enumerate}
\end{enumerate}

\section{Selbstlosigkeit}
\begin{enumerate}
	\item Der Verein ist selbstlos tätig; er verfolgt nicht in erster Linie eigenwirtschaftliche Zwecke.
	\item Mittel des Vereins dürfen nur für die satzungsgemäßen Zwecke verwendet werden. Die Mitglieder erhalten keine Gewinnanteile und in ihrer Eigenschaft als Mitglieder auch keine sonstigen Zuwendungen aus Mitteln des Vereins. Es darf keine Person durch Ausgaben, die dem Zweck des Vereins fremd sind, oder durch unverhältnismäßig hohe Vergütungen begünstigt werden.
	\item Alle Inhaber von Vereinsämtern sind ehrenamtlich tätig.
\end{enumerate}

\section{Erwerb der Mitgliedschaft}
\label{erwerb-der-mitgliedschaft}
\begin{enumerate}
	\item Mitglieder können ausschließlich natürliche Personen werden.
	\item Die Mitgliederversammlung kann Personen, die sich durch besondere Verdienste im Sinne des Vereins oder die von ihm verfolgten satzungsgemäßen Zwecke hervorgetan haben, zu Ehrenmitgliedern ernennen. Ehrenmitglieder haben alle Rechte eines ordentlichen Mitglieds. Sie sind von Beitragsleistungen befreit.
	\item Der Vorstand entscheidet auf schriftlichen Antrag des potentiellen  Mitglieds über die Aufnahme. Der Beschluss wird dem Antragsteller schriftlich oder per E-Mail mitgeteilt.
	\item Gegen den ablehnenden Bescheid des Vorstands kann der Antragsteller Beschwerde einlegen, die binnen eines Monats ab Zugang der Ablehnung schriftlich beim Vorstand einzureichen ist. Über die Beschwerde entscheidet die Mitgliederversammlung nach demselben Verfahren wie bei Ausschluss eines Mitglieds.
	\item \label{beginn-mitgliedschaft} Die Mitgliedschaft beginnt nach positivem Aufnahmebescheid mit dem Eingang des Aufnahmebeitrags und des ersten Mitgliedsbeitrags.
	\item Im Falle nicht fristgerechter Entrichtung der Beiträge ruht die Mitgliedschaft.
\end{enumerate}

\section{Beendigung der Mitgliedschaft}
Die Mitgliedschaft endet
\begin{enumerate}
	\item bei natürlichen Personen mit deren Tod.
	\item nach Austrittserklärung eines Mitglieds zum Ende des Geschäftsjahrs. Die Austrittserklärung erfordert die Schriftform und muss gegenüber dem Vorstand mit einer Frist von 6 Wochen zum Ende des Geschäftsjahres eingereicht werden.
	\item bei Mitgliedern, die sich nach schriftlicher Mahnung mehr als sechs Monate mit Mitgliedsbeiträgen im Verzug befinden.
	\item durch Ausschluss aus dem Verein.
\end{enumerate}

\section{Mitgliedsbeiträge}
\begin{enumerate}
	\item Der Verein erhebt einen Aufnahmebeitrag und einen regelmäßigen Mitgliedsbeitrag, die im Voraus zu entrichten sind. Näheres regelt eine von der Mitgliederversammlung zu beschließende Beitragsordnung.
	\item Bei Beendigung der Mitgliedschaft, gleich aus welchem Grund, erlöschen alle Ansprüche aus dem Mitgliedsverhältnis. Eine Rückerstattung von Beiträgen, Spenden oder sonstigen Unterstützungsleistungen ist grundsätzlich ausgeschlossen. Der Anspruch des Vereins auf offene Beitragsforderungen bleibt hiervon unberührt.
\end{enumerate}

\section{Organe}
Die Organe des Vereins sind
\begin{enumerate}
	\item Der Vorstand, bestehend aus dem:
	\begin{enumerate}
		\item 1. Vorsitzenden
		\item 2. Vorsitzenden
		\item Schatzmeister
	\end{enumerate}
	\item die Mitgliederversammlung.
\end{enumerate}

\section{Die Mitgliederversammlung}
\label{die-mv}
\begin{enumerate}
	\item Die Mitgliederversammlung besteht aus den Mitgliedern des Vereins.
	\item Die ordentliche Mitgliederversammlung wird einmal jährlich im ersten Halbjahr vom Vorstand einberufen.
	\item Es können außerordentliche Mitgliederversammlungen entweder auf Beschluss des Vorstands oder auf Verlangen eines Fünftels der Mitglieder einberufen werden.
	\item \label{mv-einladung} Die Einladung zur Mitgliederversammlung ist den Mitgliedern schriftlich oder per E-Mail unter Angabe von Ort, Zeit und Tagesordnung mindestens vier Wochen vorher zuzustellen. Die Einladung erfolgt an die letzte vom Mitglied bekannt gegebene Adresse.
	\item \label{mv-nachtrag} Mitglieder können zu den bestehenden Tagesordnungspunkten weitere Anträge stellen, wenn sie diese dem Vorstand spätestens zwei Wochen vor dem anberaumten Termin schriftlich oder per E-Mail zur Bekanntgabe mitteilen. Die Mitgliederversammlung beschließt über die Zulassung der nachträglichen Anträge zur Beschlussfassung.
	\item Eine Vertretung eines Mitglieds durch ein anderes ist möglich, wenn die Vertretungsbefugnis schriftlich nachgewiesen wird.
	\item Jede ordnungsgemäß einberufene Mitgliederversammlung ist unabhängig von der Zahl der erschienenen Mitglieder beschlussfähig.
	\item Die Mitgliederversammlung wird von einem anwesenden Vorstandsmitglied geleitet. Ist kein Vorstandsmitlglied anwesend, bestellt die Mitgliederversammlung einen Versammlungsleiter.
	\item{Protokollführer}
		\begin{enumerate}
			\item Die Mitglieder wählen aus ihren Reihen einen Protokollführer.
			\item Über den Verlauf der Mitgliederversammlungen ist eine Niederschrift anzufertigen, die vom Versammlungsleiter und vom Protokollführer zu unterzeichnen ist. Diese Niederschrift ist auf Anfrage beim Vorstand einsehbar. Erfolgt innerhalb von vier Wochen nach Unterzeichnung der Niederschrift kein Einspruch gilt diese als genehmigt.
			\item Die Niederschrift soll folgende Angaben enthalten:
			\begin{enumerate}
	  		\item Ort und Tag der Versammlung
	    	\item Name des Versammlungsleiters und Protokollführers
	    	\item die Zahl der erschienen Mitglieder
	    	\item Angaben zu den gefassten Beschlüssen mit genauen Abstimmungsergebnissen
	    	\item -gestrichen-
	    	\item die erforderlichen Unterschriften
			\end{enumerate}
		\end{enumerate}
	\item Jedes Mitglied, dessen Mitgliedschaft nicht ruht, ist stimmberechtigt.
\end{enumerate}

\section{Zuständigkeiten der Mitgliederversammlung}
Die Mitgliederversammlung
\begin{enumerate}
	\item wählt und kontrolliert den Vorstand.
	\item prüft und genehmigt die Jahresabschlussrechnung des Schatzmeisters und erteilt die Entlastung des Vorstands.
	\item entscheidet in allen Fällen, in denen nicht die Zuständigkeit eines anderen Organs bestimmt ist.
	\item trifft Mehrheitsentscheidungen mit der einfachen Mehrheit der teilnehmenden Mitglieder.
	\item kann den Vereinszweck mit der Zustimmung aller teilnehmenden Mitglieder ändern. Der Änderungsantrag muss gemäß \ref{die-mv}.\ref{mv-einladung} erfolgen. Weiter wird bestimmt, dass \ref{die-mv}.\ref{mv-nachtrag} für Zweckänderungen keine Anwendung findet. Zweckänderungen können somit nicht durch Nachtrag zur Tagesordunung beschlossen werden.
	\item kann die Vereinssatzung mit Zustimmung von drei Vierteln der teilnehmenden Mitglieder ändern. 
	\item gibt sich eine Geschäftsordnung.
\end{enumerate}

\section{Der Vorstand}
\begin{enumerate}
\item Der Vorstand trifft seine Beschlüsse auf Sitzungen, zu denen spätestens eine Woche vorher schriftlich oder per E-Mail zu laden ist. Mit dem Einverständnis aller Vorstandsmitglieder kann diese Frist verkürzt werden oder ganz entfallen.
	\item Der Vorstand ist beschlussfähig, wenn mindestens zwei Vorstandsmitglieder anwesend sind.
	\item Beschlüsse im Vorstand werden mit einfacher Mehrheit gefasst.
	\item Bei Ausscheiden eines Vorstandsmitglieds kann durch den Vorstand für die verbleibende Amtszeit ein Stellvertreter bestellt werden.
	\item Der Verein wird gerichtlich und außergerichtlich durch die Vorstandsmitglieder vertreten. Jeder ist alleinvertretungsberechtigt.
	\item Im Innenverhältnis wird bestimmt, dass der Vorstand in wichtigen Dingen gemeinsam beschließt.
	\item Der Vorstand wird von der Mitgliederversammlung auf die Dauer von zwei Jahren bestellt, er bleibt jedoch bis zu Bestellung eines neuen Vorstandes im Amt. Die Wiederwahl ist zulässig.
	\item Vorstandsmitglied kann nur werden, wer mindestens ein Jahr Vereinsmitglied ist und in dieser Zeit für die Ziele des Vereins förderlich tätig war. Über die Eignung des Kandidaten entscheidet die Mitgliederversammlung.
\end{enumerate}

\section{Zuständigkeiten des Vorstands}
\begin{enumerate}
	\item Der Vorstand führt die Geschäfte des Vereins und fasst die erforderlichen Beschlüsse.
	\item Er ist zu rechtsgeschäftlichen Verpflichtungen zu Lasten des Vereins bis zu einer Höhe von EUR 523,42 ermächtigt. Diese Bestimmung betrifft das Innenverhältnis.
	\item In dringenden, keinen Aufschub duldenden Dingen kann der Vorstand mit der Zustimmung aller Vorstandsmitglieder über diese Befugnisse hinaus handeln. Diese Bestimmung betrifft das Innenverhältnis. Er ist verpflichtet die Mitglieder hierüber unverzüglich zu informieren.
	\item Der Vorstand gibt sich eine Geschäftsordnung, die die Aufgabenverteilung innerhalb des Vorstands und die gegenseitige Vertretung der Vorstandsmitglieder, sowie die Art des Zustandekommens seiner Beschlüsse regelt und die der Zustimmung der Mitgliederversammlung bedarf.
\end{enumerate}

\section{Ausschluss von Mitgliedern}
\begin{enumerate}
	\item Der Vorstand kann mit einfacher Mehrheit ein Mitglied auf Antrag ausschließen.
	\item Gegen diesen Ausschluss kann schriftlich Widerspruch eingelegt werden.
	\item Ein Widerspruch führt zu einer Überprüfung des Ausschlusses durch die Mitgliederversammlung. Die einfache Mehrheit kann den Ausschluss ablehnen.
	\item Bis zur Entscheidung der Mitgliederversammlung ruht die Mitgliedschaft.
\end{enumerate}

\section{Auflösung des Vereins}
\begin{enumerate}
	\item Der Antrag auf Auflösung des Vereins kann durch den Vorstand oder ein Fünftel der Mitglieder gestellt werden.
	\item Die Auflösung des Vereines kann nur in einer eigens zu diesem Zweck einberufenen Mitgliederversammlung mit einer Mehrheit von drei Vierteln der abgegebenen gültigen Stimmen beschlossen werden. Stimmenthaltungen bleiben außer acht.
	\item Der Antrag auf Auflösung muss der Mitgliederversammlung spätestens vier
Wochen vor ihrer Tagung vorgelegt worden sein.
	\item Bei Auflösung des Vereins oder Wegfall des gemeinnützigen Zwecks fällt sein Vermögen an eine von der Mitgliederversammlung zu bestimmende,
als gemeinnützig anerkannte Körperschaft, die es zur Förderung der Volksbildung zu verwenden hat.
\end{enumerate}

\section{Ermächtigung}
\label{ermaechtigung}
Der Vorstand ist ermächtigt, etwaige zur Eintragung des Vereins und Anerkennung der Gemeinnützigkeit erforderliche formelle Änderungen und Ergänzungen der Satzung vorzunehmen.

\section{Aufwendungsersatz}
Werden Amtsträger, Mitglieder und Mitarbeiter des Vereins mit
Tätigkeiten für den Verein beauftragt, so haben sie Anrecht auf die
Erstattung ihrer Aufwendungen, sofern diese vorher dem Vorstand bekannt
gegeben wurden. Die Höhe der Erstattung erfolgt bis zur
Höhe der tatsächlich angefallenen Kosten, im Falle von Reisekosten bis
zur Höhe der billigsten Reisemöglichkeit. Die tatsächlichen
Kosten sind nachzuweisen. Abweichungen können durch den
Vorstand im Einzelfall beschlossen werden. \\[1.5cm]

\section*{Änderungen}
\begin{itemize}
	\item Die vorstehende Satzung wurde durch die Mitgliederversammlung vom 3. April 2009 neu gefasst.
	\item Durch Vorstandsbeschluss gemäß \ref{ermaechtigung} am 8. Juli 2009 geändert.
	\item Durch Beschluss der Mitgliederversammlung am 23. April 2011 geändert.
	\item Durch Beschluss der Mitgliederversammlung am 10. August 2012 geändert.
	\item Durch Beschluss der Mitgliederversammlung am 12. April 2013 geändert.
	\item Durch Beschluss der Mitgliederversammlung am 04. April 2014 geändert.
\end{itemize}
\end{document}
