\documentclass[a4paper, 12pt]{scrartcl}
\usepackage[latin1]{inputenc}
\usepackage[T1]{fontenc}
\usepackage[ngerman]{babel}
\pagestyle{plain}
 
\title{Gesch�ftsordnung der Mitgliederversammlung}
\subtitle{des Chaos Computer Club Mannheim}
\author{}
\date{tbd}

\renewcommand*\thesection{\S~\arabic{section}}

\begin{document}
\maketitle

In Ausf�llung und Erg�nzung des von der Satzung des Chaos Computer Club
Mannheim vorgegebenen Rahmens wird folgende Gesch�ftsordnung erlassen:

\section{Protokollf�hrer}
\begin{enumerate}
	\item Die Mitglieder w�hlen aus ihren Reihen einen Protokollf�hrer.
	\item �ber den Verlauf der Mitgliederversammlungen ist eine Niederschrift anzufertigen, die vom Versammlungsleiter und vom Protokollf�hrer zu unterzeichnen ist. Diese Niederschrift ist auf Anfrage beim Vorstand einsehbar. Erfolgt innerhalb von vier Wochen nach Unterzeichnung der Niederschrift kein Einspruch gilt diese als genehmigt.
	\item Die Niederschrift soll folgende Angaben enthalten:
		\begin{enumerate}
	  	\item Ort und Tag der Versammlung
	    \item Name des Versammlungsleiters und Protokollf�hrers
	    \item die Zahl der erschienen Mitglieder
	    \item Angaben zu den gefassten Beschl�ssen mit genauen Abstimmungsergebnissen
	    \item die erforderlichen Unterschriften
		\end{enumerate}
\end{enumerate}

\section{Rechnungspr�fer}
\begin{enumerate}
	\item Die Mitgliederversammlung kann, jeweils f�r die Dauer von zwei
Gesch�ftsjahren, einen Rechnungspr�fer w�hlen der nicht Mitglied des
Vorstandes ist. Eine Wiederwahl ist zul�ssig.
	\item Der Rechnungspr�fer pr�ft die Kassen- und Rechnungsf�hrung des
Vorstandes nach Ablauf eines jeden Gesch�ftsjahres und berichtet dar�ber
auf der ordentlichen Mitgliederversammlung.
	\item Die T�tigkeit ist ehrenamtlich.
	\item Der Rechnungspr�fer kann nach eigenem Ermessen unter
betriebswirtschaftlicher Beachtung der Finanzkraft des Vereins zur
Rechnungspr�fung vereidigte Wirtschaftspr�fer oder Steuerberater
hinzuziehen, welche gegebenenfalls die Kassen- und Rechnungspr�fung zu
testieren haben. Eine Verpflichtung dazu besteht nur dann, wenn die
Mitgliederversammlung dies ausdr�cklich f�r den Einzelfall beschlie�t.
\end{enumerate}

\section{Beschlussfassung und Wahlen}
\begin{enumerate}
	\item Die Abstimmung erfolgt offen, soweit keines anwesenden Mietgliedern eine geheime Abstimmung verlangt.
	\item Die offene Abstimmung erfolgt durch Handzeichen.
	\item Die Wahl des Vorstand und der Rechnungspr�fer erfolgt grunds�tzlich geheim.
\end{enumerate}

\section{Ablauf der Mitlgliederversammlung}
\begin{enumerate} 
	\item Der Versammlungsleiter er�ffnet die Sitzung und begr��t die anwesenden Mitglieder.
	\item Der Versammlungsleiter beantragt die Feststellung der Beschlussf�higkeit.
	\item Die Mitgliederversammlung bestimmt einen Protokollf�hrer.
	\item Der Versammlungsleiter beantragt die Genehmigung der Tagesordnung. Hier soll zus�tzlich dar�ber befunden werden, ob �ber nach �8, Abs.5 der Satzung gestellte Antr�ge beschossen werden kann.
	\item Die Mitgliederversammlung tritt in die Tagesordnung ein. Jeder Beschlussfassung soll eine Aussprache vorangestellt sein.
	\item Sofern ein Tagesordnungspunkt "`Verschiedenes"' existiert, soll dieser nur f�r Informationen und Ank�ndigungen verwendet werden. Innerhalb dieses Tagesordnungspunktes sollen keine Beschl�sse gefasst werden.
\end{enumerate}

\section{Inkrafttreten}
Diese Gesch�ftsordnung wurde durch die Mitgliederversammlung vom tbd
beschlossen und tritt mit sofortiger Wirkung in Kraft.\\
Diese Gesch�ftsordnung ersetzt alle vorher beschlossenen Gesch�ftsordnungen.

\end{document}
