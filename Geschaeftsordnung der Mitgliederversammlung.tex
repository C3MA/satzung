\documentclass[a4paper, 12pt]{scrartcl}
\usepackage[utf8]{inputenc}
\usepackage[T1]{fontenc}
\usepackage[ngerman]{babel}
\pagestyle{plain}
 
\title{Geschäftsordnung der Mitgliederversammlung}
\subtitle{des Chaos Computer Club Mannheim e.V.}
\author{}
\date{7. August 2015}

\renewcommand*\thesection{\S~\arabic{section}}

\begin{document}
\maketitle

\noindent In Ausfüllung und Ergänzung des von der Satzung des Chaos Computer Club
Mannheim e.V. vorgegebenen Rahmens wird folgende Geschäftsordnung erlassen:

\section{Protokollführer}
-gestrichen-

\section{Rechnungsprüfer}
\begin{enumerate}
	\item Die Mitgliederversammlung kann, jeweils für die Dauer von zwei
Geschäftsjahren, einen Rechnungsprüfer wählen, der nicht Mitglied des
Vorstandes ist. Eine Wiederwahl ist zulässig.
	\item Der Rechnungsprüfer prüft die Kassen- und Rechnungsführung des
Vorstandes nach Ablauf eines jeden Geschäftsjahres und berichtet darüber
auf der ordentlichen Mitgliederversammlung.
	\item Die Tätigkeit ist ehrenamtlich.
	\item Der Rechnungsprüfer kann nach eigenem Ermessen unter
betriebswirtschaftlicher Beachtung der Finanzkraft des Vereins zur
Rechnungsprüfung vereidigte Wirtschaftsprüfer oder Steuerberater
hinzuziehen, welche gegebenenfalls die Kassen- und Rechnungsprüfung zu
testieren haben. Eine Verpflichtung dazu besteht nur dann, wenn die
Mitgliederversammlung dies ausdrücklich für den Einzelfall beschließt.
\end{enumerate}

\section{Beschlussfassung und Wahlen}
\begin{enumerate}
	\item Die Abstimmung erfolgt offen, soweit keines der anwesenden Mitgliedern eine geheime Abstimmung verlangt.
	\item Die offene Abstimmung erfolgt durch Handzeichen.
	\item Die Wahl des Vorstands und des Rechnungsprüfers erfolgt grundsätzlich geheim.
\end{enumerate}

\section{Ablauf der Mitgliederversammlung}
\begin{enumerate} 
	\item Der Vorstand eröffnet die Sitzung und begrüßt die anwesenden Mitglieder.
	\item Die Mitgliederversammlung bestimmt einen Versammlungsleiter und einen Protokollführer.
	\item Der Versammlungsleiter stellt die Beschlussfähigkeit fest.
	\item Der Versammlungsleiter beantragt die Genehmigung der Tagesordnung. Hier soll zusätzlich darüber befunden werden, ob über nach §8, Abs.5 der Satzung gestellte Anträge beschossen werden kann.
	\item Die Mitgliederversammlung tritt in die Tagesordnung ein. Jeder Beschlussfassung soll eine Aussprache vorangestellt sein.
	\item Sofern ein Tagesordnungspunkt "`Verschiedenes"' existiert, soll dieser nur für Informationen und Ankündigungen verwendet werden. Innerhalb dieses Tagesordnungspunktes sollen keine Beschlüsse gefasst werden.
\end{enumerate}

\section{Inkrafttreten}
Diese Geschäftsordnung wurde durch die Mitgliederversammlung vom 7. August 2015 beschlossen und tritt mit sofortiger Wirkung in Kraft.\\
Diese Geschäftsordnung ersetzt alle vorher beschlossenen Geschäftsordnungen.

\end{document}
