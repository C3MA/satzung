\documentclass[a4paper, 12pt]{scrartcl}
\usepackage[latin1]{inputenc}
\usepackage[T1]{fontenc}
\usepackage[ngerman]{babel}
\pagestyle{plain}
 
\title{Gesch�ftsordnung des Vorstands}
\subtitle{des Chaos Computer Club Mannheim}
\author{}
\date{9. Mai 2014}

\renewcommand*\thesection{\S~\arabic{section}}

\begin{document}
\maketitle

\section{Versammlungsordnung}
\begin{enumerate}
	\item Der Vorstand soll einmal im Quartal tagen.
	\item Der Vorstand w�hlt aus seinen Reihen einen Protokollf�hrer, der den Ablauf der Vorstandssitzung protokolliert.
	\item �ber den Verlauf der Vorstandssitzungen ist eine Niederschrift anzufertigen, die vom allen Anwesenden zu unterzeichnen ist. Die Niederschrift ist innerhalb einer Woche den Mitgliedern schriftlich oder per E-Mail zur Verf�gung zu stellen. Erfolgt nach der Ver�ffentlichung der Niederschrift innerhalb von vier Wochen kein Einspruch, gilt diese als genehmigt.
\end{enumerate}

\section{Berichtspflicht des Vorstands}
\begin{enumerate}
	\item Mit dem Ablauf des Gesch�ftsjahres
		\begin{enumerate}
			\item erstellt der Vorstand den Gesch�ftsbericht
			\item  stellt der Schatzmeister unverz�glich die Abrechnung sowie die Verm�gens�bersicht und sonstige Unterlagen von wirtschaftlichem Belang dem Rechnungspr�fer des Vereins zur Pr�fung zur Verf�gung.
		\end{enumerate}
\end{enumerate}

\section{Beir�te}
Der Vorstand kann "`Fachliche Beir�te"' oder "`Wissenschaftliche Beir�te"' einrichten,
die f�r den Verein beratend und unterst�tzend t�tig werden; in die Beir�te k�nnen
auch Nicht-Mitglieder berufen werden.

\section{Inkrafttreten}
Diese Gesch�ftsordnung wurde durch den Vorstand am 9. Mail 2014 beschlossen.\\
Sie wird der Mitgliederversammlung zur Genehmigung in der n�chsten Mitgliederversammlung vorgelegt.\\
Diese Gesch�ftsordnung ersetzt alle vorher beschlossenen Gesch�ftsordnungen.
\end{document}
